\documentclass[sigplan]{acmart}

\usepackage{amsmath}
\usepackage{mathpartir}

\usepackage{tikz}
\usetikzlibrary{arrows,automata}

\usepackage{enumerate}
\usepackage{xspace}
\usepackage[capitalise]{cleveref}
\usepackage{caption}
\usepackage{subcaption}
\usepackage{graphicx}
\usepackage{booktabs}
\usepackage{array}
\usepackage{rotating}
\usepackage{xcolor}

\newcommand\hide[1]{}

%% Bibliography style
\bibliographystyle{ACM-Reference-Format}
\citestyle{acmauthoryear}

\usepackage{todonotes}
\newcommand{\anshuman}[2][]{\todo[#1,backgroundcolor=blue!20]{A:\@ #2}}

\begin{document}

\title[Explain it Like I'm Five]{Explain it Like I'm Five: How I Taught Ms Zdan's First-Graders About Ariane 5, and You Can Too}

\author{Anshuman Mohan}
%\orcid{nnnn-nnnn-nnnn-nnnn}
\affiliation{
  \department{Computer Science}
  \institution{Cornell University}
  \city{Ithaca}
  \state{New York}
  \postcode{14853-7501} % chktex 8
  \country{USA}
}
\email{amohan@cs.cornell.edu}

\begin{abstract}

Little kids use computers all the time.
The phone is the new pacifier; the iPad is the new slate and chalk.
Increasingly early in their schooling, kids also learn how to code.
We must scare them straight by teaching them formal methods before all else.

Over four half-hour sessions, I taught a class of first-graders enough computer science to understand and appreciate the Ariane 5 space shuttle disaster.
Here's how.

\end{abstract}

\maketitle

\section*{Lesson 1: A tale of two hungry cities}

I went into class with eight pieces of card, each colored red on one side and blue on the other.
I split the class into two groups, standing in huddles on opposite ends of the room.
I handed each group one card and explained the game:

``You are two cities on hills, and you have a zipline that carries food between you.
Occasionally you'd like to order something to eat from the other city, but the cities are rather far apart and you can only communicate using your cards.
A card is either red or blue, and you can order either a burger or spaghetti.''

I allowed them one meeting to come up with a plan, and they quickly had something: red would mean a burger, and blue spaghetti.

\begin{table}[h]\sffamily
  \begin{tabular}{cc}
  Signal & Order \\
  \midrule
  R & burger \\
  B & spaghetti \\
  \end{tabular}
  % \caption{Iteration 1.}%
\end{table}

I sneaked up to one city and whispered, ``Can you order me a burger, please?'', and the students delivered.
I ran to the other city and got them to order me another burger and some spaghetti.
Then I went to the first city and requested that they order me a hot dog.

Pandemonium.
I was accused of being a naughty man.

A student proposed that \emph{no card} could mean a hot dog, but I steered them away from that: ``Well some days you may want to order nothing at all; reserve no card for that.''

Another proposed balancing the card between red and blue, or swapping quickly between red and blue; I steered them clear of these too.
After a minute I offered each city a second card, and they had an answer right away.
The old rules would remain, and \emph{any} two cards would mean hot dog.

\begin{table}[h]\sffamily
  \begin{tabular}{cc}
  Signal & Order \\
  \midrule
  R & burger \\
  B & spaghetti \\
  RR, BB, BR, RB & hot dog
  \end{tabular}
  % \caption{Iteration 2.}%
\end{table}

This made me a little nervous, since it was not part of the plan, but I steered things along:
``Well this is great, but what if the other city confuses the hot dog order with just a double order, or a burger that is chopped up and mixed into spaghetti?''

They figured out the way to make me happy:

\begin{table}[h]\sffamily
  \begin{tabular}{cc}
  Signal & Order \\
  \midrule
  RR & burger \\
  BB & spaghetti \\
  RB, BR & hot dog
  \end{tabular}
  % \caption{Iteration 3.}%
\end{table}

\noindent and on prodding, they realized they could say four things:

\begin{table}[h]\sffamily
  \begin{tabular}{cc}
  Signal & Order \\
  \midrule
  RR & burger \\
  BB & spaghetti \\
  RB & hot dog \\
  BR & bibimbap
  \end{tabular}
  % \caption{Iteration 4.}%
\end{table}

Finally I handed them a third card, asked them to make the longest list they could, and left the room.
I returned to a menu of eight items and some rather hungry kids.

We were out of time, but the students had enough fun that Ms Zdan requested that I leave the cards for them to play with until next week's lesson.
I left her with all eight cards, and a hint: that $n$ cards can say $2^n$ things.

\section*{Lesson 2: How a computer stores a number}

TK, but the punchline is that after this lesson we got to:

\begin{table}[h]\sffamily
  \begin{tabular}{ccc}
  Signal (cards) & Signal (binary) & Number (decimal) \\
  \midrule
  RRR & 000 & 0 \\
  RRB & 001 & 1 \\
  RBR & 010 & 2 \\
  RBB & 011 & 3 \\
  BRR & 100 & 4 \\
  BRB & 101 & 5 \\
  BBR & 110 & 6 \\
  BBB & 111 & 7 \\
  \end{tabular}
  % \caption{Iteration 4.}%
\end{table}

\section*{Lesson 3: A look at a circuit, and the end of Moore's law}

This was a digression.
I brought in an FPGA and showed them the wires running between components.
I drew the analogy to the game.
The students were super interested in the fan, however, and we entered into an unplanned discussion about Moore's law.

Fuller description TK.

\section*{Lesson 4: Why $8$ looks like $0$}

TK, but here's a sketch.

I asked for a volunteer and we stood together in front of the class.
We were at $0$ on a number line, and moving to our right meant adding $1$.
She was a human, and I a computer.
We counted from $0$ through $7$ no problem, but on adding $1$ to $7$ she took a step to her right and I ran all the way back to $0$.

I mapped this back to regular math, where performing $9 + 1 = 10$ requires an additional digit.
What do you do when you are not allowed that additional digit?
There isn't a clean answer, but you're going to have to do \emph{something}.
We went over the table that we had already drawn up and explored some of the binary addition that was implicit.
\begin{itemize}
  \item
  Adding one to $1_{dec}$ is the same as adding one to $001_{bin}$.
  How do we do it? We carry the $1$, just like we do in regular math, and arrive at $010_{bin} = 2_{dec}$.

  \item
  Adding one to $3_{dec}$ is the same as adding one to $011_{bin}$.
  It's a little more complicated than last time---a little like $99+1 = 100$ in regular math---but we figure it out and arrive at $100_{bin} = 4_{dec}$.

  \item
  Adding one to $7_{dec}$ is the same as adding one to $111_{bin}$, and \emph{we} know how to do that and arrive at $1000_{bin} = 8_{dec}$.
  However, a computer that has just three cards is in trouble: it does the math, gets to an answer, and then realizes that it doesn't have enough cards to write down the answer.
  It drops the leftmost digit.
  We know that
  $$7_{dec} + 1_{dec} = 111_{bin} + 1_{bin} = 1000_{bin} = 8_{dec},$$
  but a computer performs
  $$7_{dec} + 1_{dec} = 111_{bin} + 1_{bin} = {\color{red}1}000_{bin} = 0_{dec}.$$
  Uh oh.
\end{itemize}

I mapped this back to overflow, and why it is such a big issue.
If you're in your car with your mom and you \emph{think} you're doing $0$ miles per hour, but you're actually doing $8$, you may be in trouble.
Eight miles an hour is not very fast, but there's a whole bunch of stuff that we're allowed to do at $0$ miles an hour that we're not allowed to do at $8$ miles an hour.
At $0$ miles an hour, you can unfasened your seatbelt, lean over the seat, and even get out of the car and walk around.

This is what happened with the Ariane 5 rocket.
It was in fact flying super fast away from the earth, but the on-board computer thought it was going backwards towards the earth.
The computer triggered a self-destruct sequence because it thought it was going to crash into the earth.


\section*{Reflections}

TK

\section*{Acknowledgments}

My time at Beverly J Martin Elementary School was facilitated by Cornell University's \textsc{grasshopr} program, a volunteer organization that pairs graduate students with K-12 teachers so we can share our research with curious students.
This is hard, important, rewarding work, and you are superstars for making it happen every year.

An enormous thank you to Ms Lisa Zdan, her teaching aides A and B\anshuman{ask}, and of course her students.

\appendix
\section{Beyond Code: How Computers Talk in 0s and 1s}
\label{appendix:pitch}

\emph{What follows is the pitch I originally submitted to the \textsc{grasshopr} program when requesting to be paired with a teacher.
It is aimed at middle-schoolers, with options for dialing it up or down for other ages.}

Computers don't speak English, but they don't quite speak Python either.
Under the hood of every computer is a vast troupe of $0$s and $1$s doing a carefully orchestrated dance.
This language of $0$s and $1$s is called binary.
In this mini-course, I aim to reveal to students the ``wonder of the everyday'' inside our computers.

Below is a lesson-by-lesson plan for students in middle school; I am very happy to adjust this for other ages.

\begin{enumerate}
  \item \emph{Why do computers speak binary at all? A historical and mechanical perspective.}

  This lesson will explain the origins of the binary system to students.
  As an activity, I will invite students to ``say as much as possible'' using flags that they can raise or lower.
  First one flag (two possible states: up or down), then two flags (four states: up/up, down/down, up/down, and down/up), then three flags (eight states), and so on.
  I will explain how this translates to computers from a mechanical engineering perspective.

  \item \emph{The rules of the game: addition, multiplication, and encoding, all in binary.}

  This lesson will get into the mathematical foundations of binary.
  How do we do arithmetic on these strange numbers?
  What is different compared to normal arithmetic, and what is, remarkably, the same?
  This will be a more traditional math lesson, but perhaps with an artsy activity to serve as a pedagogical tool.
  By the end of this lesson, students will be comfortable doing basic math with binary numbers and translating back and forth between binary numbers and regular decimal numbers.

  \item \emph{What happens when you count to infinity? Overflow, and why it hurts (from my own research, published in July '21).}

  Computers cannot count forever, and run into problems when they try.
  This lesson will explain the fascinating mechanism of overflow, whereby a computer misrepresents a very large number as a very small number.
  As an activity, we will count to infinity (in binary) as a class, and feel for ourselves the confusion of overflow.
  I will also present a recent research finding where overflow occurs in a very unexpected and ubiquitous place.

  \item \emph{Compilation: the journey from English to Python to binary.}

  If we plan our code in English and speak to computers in languages such as Python, Java, and C, how do our instructions get translated into 0s and 1s?
  This is the work of \emph{compilers}, and these are the focus of this lesson.
  As an activity, we will ``compile'' a Wikipedia article from English to Simple English (e.g. https://en.wikipedia.org/wiki/ Barack\_Obama to https://simple.wikipedia.org/wiki/Barack\_Obama) and observe the kinds of changes we make.

  \item \emph{Quantum computing: a future without binary?}

  Binary has long been the language of computers, but exciting new developments in quantum computing may change this.
  This lesson will give a brief overview of the physical and mechanical obstacles facing quantum computers.
  We will also explore the amazing advances they promise, and how they could change computing and security forever.

\end{enumerate}

For students in elementary school, I would split lesson 2 into two lessons, split lesson 3 into two lessons, and drop lessons 4 and 5.
For students in high school, I would combine lessons 1 and 2, and expand lesson 3 into a slightly deeper exploration of my research finding.

\bibliography{bibliography}

\end{document}
