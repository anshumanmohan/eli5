\section{Beyond Code: How Computers Talk in 0s and 1s}
\label{appendix:pitch}

\emph{What follows is the pitch I originally submitted to the \textsc{grasshopr} program when requesting to be paired with a teacher.
It is aimed at middle-schoolers, with options for dialing it up or down for other ages.}

Computers don't speak English, but they don't quite speak Python either.
Under the hood of every computer is a vast troupe of $0$s and $1$s doing a carefully orchestrated dance.
This language of $0$s and $1$s is called binary.
In this mini-course, I aim to reveal to students the ``wonder of the everyday'' inside our computers.

Below is a lesson-by-lesson plan for students in middle school; I am very happy to adjust this for other ages.

\begin{enumerate}
  \item \emph{Why do computers speak binary at all? A historical and mechanical perspective.}

  This lesson will explain the origins of the binary system to students.
  As an activity, I will invite students to ``say as much as possible'' using flags that they can raise or lower.
  First one flag (two possible states: up or down), then two flags (four states: up/up, down/down, up/down, and down/up), then three flags (eight states), and so on.
  I will explain how this translates to computers from a mechanical engineering perspective.

  \item \emph{The rules of the game: addition, multiplication, and encoding, all in binary.}

  This lesson will get into the mathematical foundations of binary.
  How do we do arithmetic on these strange numbers?
  What is different compared to normal arithmetic, and what is, remarkably, the same?
  This will be a more traditional math lesson, but perhaps with an artsy activity to serve as a pedagogical tool.
  By the end of this lesson, students will be comfortable doing basic math with binary numbers and translating back and forth between binary numbers and regular decimal numbers.

  \item \emph{What happens when you count to infinity? Overflow, and why it hurts (from my own research, published in July '21).}

  Computers cannot count forever, and run into problems when they try.
  This lesson will explain the fascinating mechanism of overflow, whereby a computer misrepresents a very large number as a very small number.
  As an activity, we will count to infinity (in binary) as a class, and feel for ourselves the confusion of overflow.
  I will also present a recent research finding where overflow occurs in a very unexpected and ubiquitous place.

  \item \emph{Compilation: the journey from English to Python to binary.}

  If we plan our code in English and speak to computers in languages such as Python, Java, and C, how do our instructions get translated into 0s and 1s?
  This is the work of \emph{compilers}, and these are the focus of this lesson.
  As an activity, we will ``compile'' a Wikipedia article from English to Simple English (e.g. https://en.wikipedia.org/wiki/ Barack\_Obama to https://simple.wikipedia.org/wiki/Barack\_Obama) and observe the kinds of changes we make.

  \item \emph{Quantum computing: a future without binary?}

  Binary has long been the language of computers, but exciting new developments in quantum computing may change this.
  This lesson will give a brief overview of the physical and mechanical obstacles facing quantum computers.
  We will also explore the amazing advances they promise, and how they could change computing and security forever.

\end{enumerate}

For students in elementary school, I would split lesson 2 into two lessons, split lesson 3 into two lessons, and drop lessons 4 and 5.
For students in high school, I would combine lessons 1 and 2, and expand lesson 3 into a slightly deeper exploration of my research finding.